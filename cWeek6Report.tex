\documentclass[11pt]{article}

\usepackage{a4wide,times}
\usepackage[english]{babel}

% -----------------------------------------------
% especially use this for you code
% -----------------------------------------------

\usepackage{courier}
\usepackage{listings}
\usepackage{color}
\usepackage{tabularx}

\definecolor{Gray}{gray}{0.95}

\definecolor{mygreen}{rgb}{0,0.6,0}
\definecolor{mygray}{rgb}{0.5,0.5,0.5}
\definecolor{mymauve}{rgb}{0.58,0,0.82}

\lstset{language=C++,
	basicstyle = \normalsize\ttfamily,   % the size and fonts that are used
	tabsize = 2,                    % sets default tabsize
	breaklines = true,              % sets automatic line breaking
	keywordstyle=\color{blue}\ttfamily,
	stringstyle=\color{red}\ttfamily,
	commentstyle=\color{mygreen}\ttfamily,
	numbers=left,
	keepspaces=true,
	showspaces=false,
	showstringspaces=false,
}

\lstdefinestyle{bigtabs}
{
	basicstyle = \normalsize\ttfamily,
	tasize = 8;
}

\begin{document}

\title{Programming in C/C++ \\
       Exercises set six: Basic Input/Output
}
\date{\today}
\author{Christiaan Steenkist \\
Diego Ribas Gomes \\
Jaime Betancor Valado \\
Remco Bos \\
}

\maketitle

\section*{Exercise 51, understanding the behaviour of istreams}

A piece of code is presented which is expected to take and output two numerical unsigned inputs, but outputs 0 instead of the second value. The reason for this is that the stream's data hasn't been activated by a seek operation after receiving the value by the $<$$<$ operator.
The fix is shown in line 3 of the following listing:


\lstinputlisting[caption = code repair]{a51.cc}

\section*{Exercise 52, defining a manipulator}

A manipulator is implemented to insert the current date and time as produced by asctime but without the trailing newline it automatically appends.

\lstinputlisting[caption = nowManip.cc]{a52.cc}

\section*{Exercise 53, displaying floating point numbers using modifiers}

A program is built that defines a double variable and displays it in different requested formats using a single cout statement.

\lstinputlisting[caption = floatModif.cc]{a53.cc}

\section*{Exercise 54, using binary files}

A program is built reading by default the binary file /var/log/account/pacct or another specified command-line argument and outputs the names of all processes that didn't exit properly. The inclusion of a '-a' option should make it output information about all exited processes and if a process was killed with SIGKILL or SIGTERM, it should mention the name of the signal instead of the number.

\lstinputlisting[caption = main.cc]{a54/main.cc}
\lstinputlisting[caption = main.h]{a54/main.h}
\lstinputlisting[caption = main.ih]{a54/main.ih}
\lstinputlisting[caption = arguments.cc]{a54/arguments.cc}
\lstinputlisting[caption = process.cc]{a54/process.cc}
\lstinputlisting[caption = printprocess.cc]{a54/printprocess.cc}

\end{document}